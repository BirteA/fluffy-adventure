\documentclass[10pt,]{article}
\usepackage[left=1in,top=1in,right=1in,bottom=1in]{geometry}
\newcommand*{\authorfont}{\fontfamily{phv}\selectfont}
\usepackage[]{mathpazo}


  \usepackage[T1]{fontenc}
  \usepackage[utf8]{inputenc}


\providecommand{\tightlist}{%
  \setlength{\itemsep}{0pt}\setlength{\parskip}{0pt}}

\usepackage{abstract}
\renewcommand{\abstractname}{}    % clear the title
\renewcommand{\absnamepos}{empty} % originally center

\renewenvironment{abstract}
 {{%
    \setlength{\leftmargin}{0mm}
    \setlength{\rightmargin}{\leftmargin}%
  }%
  \relax}
 {\endlist}

\makeatletter
\def\@maketitle{%
  \newpage
%  \null
%  \vskip 2em%
%  \begin{center}%
  \let \footnote \thanks
    {\fontsize{18}{20}\selectfont\raggedright  \setlength{\parindent}{0pt} \@title \par}%
}
%\fi
\makeatother




\setcounter{secnumdepth}{0}

\usepackage{longtable,booktabs}


\usepackage{graphicx}


\title{MTXQCvX - Experimental Setup - PROJECT TITLE \thanks{Kempa Lab - MTXQCvX ExperimentalSetup, provided by Ch. Zasada, processed
`September 25, 2018'}  }



\author{\Large NAME ONE\vspace{0.05in} \newline\normalsize\emph{LAB, LAB LOCATION}   \and \Large NAME TWO\vspace{0.05in} \newline\normalsize\emph{LAB, LAB LOCATION}  }


\date{}

\usepackage{titlesec}

\titleformat*{\section}{\normalsize\bfseries}
\titleformat*{\subsection}{\normalsize\itshape}
\titleformat*{\subsubsection}{\normalsize\itshape}
\titleformat*{\paragraph}{\normalsize\itshape}
\titleformat*{\subparagraph}{\normalsize\itshape}


\usepackage{natbib}
\bibliographystyle{apsr}



\newtheorem{hypothesis}{Hypothesis}
\usepackage{setspace}

\makeatletter
\@ifpackageloaded{hyperref}{}{%
\ifxetex
  \usepackage[setpagesize=false, % page size defined by xetex
              unicode=false, % unicode breaks when used with xetex
              xetex]{hyperref}
\else
  \usepackage[unicode=true]{hyperref}
\fi
}
\@ifpackageloaded{color}{
    \PassOptionsToPackage{usenames,dvipsnames}{color}
}{%
    \usepackage[usenames,dvipsnames]{color}
}
\makeatother
\hypersetup{breaklinks=true,
            bookmarks=true,
            pdfauthor={NAME ONE (LAB, LAB LOCATION) and NAME TWO (LAB, LAB LOCATION)},
             pdfkeywords = {MTXQCvX, GC-MS, metabolomics, data analysis and processing},  
            pdftitle={MTXQCvX - Experimental Setup - PROJECT TITLE},
            colorlinks=true,
            citecolor=blue,
            urlcolor=blue,
            linkcolor=magenta,
            pdfborder={0 0 0}}
\urlstyle{same}  % don't use monospace font for urls



\begin{document}
	
% \pagenumbering{arabic}% resets `page` counter to 1 
%
% \maketitle

{% \usefont{T1}{pnc}{m}{n}
\setlength{\parindent}{0pt}
\thispagestyle{plain}
{\fontsize{18}{20}\selectfont\raggedright 
\maketitle  % title \par  

}

{
   \vskip 13.5pt\relax \normalsize\fontsize{11}{12} 
\textbf{\authorfont NAME ONE} \hskip 15pt \emph{\small LAB, LAB LOCATION}   \par \textbf{\authorfont NAME TWO} \hskip 15pt \emph{\small LAB, LAB LOCATION}   

}

}



{
\hypersetup{linkcolor=black}
\setcounter{tocdepth}{2}
\tableofcontents
}




\begin{abstract}

    \hbox{\vrule height .2pt width 39.14pc}

    \vskip 8.5pt % \small 

\noindent This document provides an evaluation of GC-MS derived metabolomics data.
It asseses GC-MS performance, the absolute quantification and the stable
isotope incorporation. ADD HERE FURTHER PROJECT RELEVANT FACTS.


\vskip 8.5pt \noindent \emph{Keywords}: MTXQCvX, GC-MS, metabolomics, data analysis and processing \par

    \hbox{\vrule height .2pt width 39.14pc}



\end{abstract}


\vskip 6.5pt

\noindent  \section{Project-related experimental
setup}\label{project-related-experimental-setup}

Sample extraction and derivatisation have been performed by Jenny.
According to her notes documented in OneNote section:
Collaborations/Landthaler/ODC1 experiment

\section{Sample extraction}\label{sample-extraction}

Extraction protocol:

\begin{verbatim}
  1. 5 ml MeOH (50%)
  2. 1 ml Chlorform
  3. dried 3.5 ml of polar phase
  4. 2nd extraction: yes (in two replicates)
\end{verbatim}

Polar phases have been split into two equal fractions of \emph{280} ul
(added 600 ul 20\% MeOH).

\section{Quant-Mix extraction
protocol}\label{quant-mix-extraction-protocol}

Quant-Mixes batch: \emph{Quant\_v4}

\begin{verbatim}
   1. 1 ml MCW for extraction
   2. 0.5 ml H2O for phase separation
   3. dried 0.5 ml of polar phase (twice)
\end{verbatim}

\section{Derivatisation protocol}\label{derivatisation-protocol}

Applied the following protocol for derivatisation protocol

\begin{verbatim}
  1. MEOX/Pyridine (final conc: 40 mg MEOX/ 1 ml Pyridine)
    - Volume: 20 ul
    - Incubation time: 90 min
    - Temp: 30 C
  2. Alkan-mix/MSTFA (10 ul mix/1 ml MSTFA)
    - Volume: 80 ul
    - Incubation time: 60 min
    - Temp: 37 C
    
\end{verbatim}

Prepared aliquots: three-times 28 ul, big glas vials, crimped.

\section{GC-MS measurement}\label{gc-ms-measurement}

Samples have been measured using the following methods

\begin{verbatim}
  1. Injector-method: hamilton_1ul
  2. GC-method: 5/7/12 1.2ml/min
  3. Split: 1:5
  4. MS-method: Lizzy-like
  
\end{verbatim}

\section{General MTXQC parameter}\label{general-mtxqc-parameter}

\begin{verbatim}
## MTXQC_params.csv written.
\end{verbatim}

\begin{verbatim}
## Proceed with MTXQC_metmax in order to generate required input files.
\end{verbatim}

\begin{longtable}[]{@{}ll@{}}
\caption{Experimental parameters of the project.}\tabularnewline
\toprule
Value & Parameter\tabularnewline
\midrule
\endfirsthead
\toprule
Value & Parameter\tabularnewline
\midrule
\endhead
test & subf\tabularnewline
annotation.csv & ann\tabularnewline
Sample\_extracts.csv & sample\_ext\tabularnewline
TRUE & instd\tabularnewline
Quant1\_v3 & quant\tabularnewline
glc & substr\tabularnewline
pSIRM & data\tabularnewline
500 & quant\_vol\tabularnewline
no & addQ\_Int\tabularnewline
no & addQ\tabularnewline
metmax & inputformat\tabularnewline
1 & backups\tabularnewline
cell extracts & samples\tabularnewline
\bottomrule
\end{longtable}
\newpage
\singlespacing 
\end{document}